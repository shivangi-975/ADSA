\documentclass[12pt]{article}%
\usepackage{amsfonts}
\usepackage{fancyhdr}
\usepackage{comment}
\usepackage[a4paper, top=2.5cm, bottom=2.5cm, left=2.2cm, right=2.2cm]%
{geometry}
\usepackage{times}
\usepackage{amsmath}
\usepackage{changepage}
\usepackage{amssymb}
\usepackage{graphicx}%
\setcounter{MaxMatrixCols}{30}
\newtheorem{theorem}{Theorem}
\newtheorem{acknowledgement}[theorem]{Acknowledgement}
\newtheorem{algorithm}[theorem]{Algorithm}
\newtheorem{axiom}{Axiom}
\newtheorem{case}[theorem]{Case}
\newtheorem{claim}[theorem]{Claim}
\newtheorem{conclusion}[theorem]{Conclusion}
\newtheorem{condition}[theorem]{Condition}
\newtheorem{conjecture}[theorem]{Conjecture}
\newtheorem{corollary}[theorem]{Corollary}
\newtheorem{criterion}[theorem]{Criterion}
\newtheorem{definition}[theorem]{Definition}
\newtheorem{example}[theorem]{Example}
\newtheorem{exercise}[theorem]{Exercise}
\newtheorem{lemma}[theorem]{Lemma}
\newtheorem{notation}[theorem]{Notation}
\newtheorem{problem}[theorem]{Problem}
\newtheorem{proposition}[theorem]{Proposition}
\newtheorem{remark}[theorem]{Remark}
\newtheorem{solution}[theorem]{Solution}
\newtheorem{summary}[theorem]{Summary}
\newenvironment{proof}[1][Proof]{\textbf{#1.} }{\ \rule{0.5em}{0.5em}}

\newcommand{\Q}{\mathbb{Q}}
\newcommand{\R}{\mathbb{R}}
\newcommand{\C}{\mathbb{C}}
\newcommand{\Z}{\mathbb{Z}}

\begin{document}

\title{Assignment 1}
\author{Shivangi Parashar}
\date{\today}
\maketitle

\section{Lines and Planes}
 
Q.1) For which values of a and b does the following pair of linear equations have infinite number of solutions?\\
\\
(2 3 )x =7 \\
(a-b   a+b)x=3a+b-2
\\
\\
{Solution:}
we are given that linear equation has infinite number of solutions and we need to find values of a and b\\
\\
2x+3y=7\\
2x+3y-7=0....................................eqn 1\\
(a-b)x+(a+b)y=3a+b−2\\
(a-b)x+(a+b)y-3a+b−2=0....................eqn 2\\
The system of an equation has infinitely many solutions when the \underline{lines are coincident}, and they have the same \underline{y-intercept}. \\
If the two lines have the y-intercept and the slope, they are actually in the same exact line then the system should have infinite solutions.\\

a1 / a2 = b1 / b2 = c1 / c2 ..............eqn 3\\
\\
For  the pair of lines to be  coincident and the pair of equations is dependent and consistent.\\
Considering eqn1 we have :\\
\\
2x+3y-7=0....................................eqn 1
\\
From the above eqn 1 we have :\\
\\
 a1=2 \\
 b1=3\\
 c1=-7\\
(a-b)x+(a+b)y-3a+b−2=0....................eqn 2\\
From the above eqn 2 we have :
\\
a2=(a-b)\\
b2=(a+b)\\
c2=-(3a+b-2)\\
\\
\\
Putting values of a1,b1,c1 and a2,b2,c2 in eqn 3 we have.
\\
2/a-b= 3/a+b=-7/-(3a+b-2)..................eqn 4
\\
Solving 2/(a-b)=3/(a+b) from eqn 4\\
\\
2b+ 3b=3a-2a
\\
a=5b........................................eqn 5
\\
Solving 2/(a-b)=7/(3a+b-2) from eqn 4
\\
9b-4=a\\
\\
putting a=5b from eqn 5 we have:\\
9b-4=5b
\\
\\
b=1
Now putting b in eqn 5 we have:\\
a=5
\\
\\
Hence value of \underline{a =5 and b=1 }for which above eqn will have infinite solutions.
\\
\\
Now if we put the values of a and b and if the ranks of augmented and coefficient matrix are equal but less than the full rank for that values of a and b than we can say that our equation has infinite solution









\end{document}